\documentclass[]{article}

% Imported Packages
%------------------------------------------------------------------------------
\usepackage{amssymb}
\usepackage{amstext}
\usepackage{amsthm}
\usepackage{amsmath}
\usepackage{enumerate}
\usepackage{fancyhdr}
\usepackage[margin=1in]{geometry}
\usepackage{graphicx}
%\usepackage{extarrows}
\usepackage{setspace}
\usepackage{float}
\usepackage{array}
%------------------------------------------------------------------------------

% Header and Footer
%------------------------------------------------------------------------------
\pagestyle{plain}  
\renewcommand\headrulewidth{0.4pt}                                      
\renewcommand\footrulewidth{0.4pt}                                    
%------------------------------------------------------------------------------

% Title Details
%------------------------------------------------------------------------------
\title{Deliverable \#3 Template}
\author{SE 3A04: Software Design II -- Large System Design}
\date{}                               
%------------------------------------------------------------------------------

% Document
%------------------------------------------------------------------------------
\begin{document}

\maketitle	

\section{Introduction}
\label{sec:introduction}
% Begin Section

This section should provide an brief overview of the entire document.

\subsection{Purpose}
\label{sub:purpose}
% Begin SubSection
\begin{enumerate}[a)]
	\item Delineate the purpose of the document
	\item Specify the intended audience for the document
\end{enumerate}
% End SubSection

\subsection{System Description}
\label{sub:system_description}
% Begin SubSection
\begin{enumerate}[a)]
	\item Give a brief description of the system. This could be a paragraph or two to give some context to this document.
\end{enumerate}
% End SubSection

\subsection{Overview}
\label{sub:overview}
% Begin SubSection
\begin{enumerate}[a)]
	\item Describe what the rest of the document contains 
	\item Explain how the document is organised
\end{enumerate}

% End SubSection

% End Section

\section{State Charts for Controller Classes}
\label{sec:state_charts_for_controller_classes}
% Begin Section
This section should provide a state chart for each controller class for your application.
% End Section

\section{Sequence Diagrams}
\label{sec:sequence_diagrams}
% Begin Section
This section should provide a sequence diagram for each use case of your application.
% End Section

\section{Detailed Class Diagram}
\label{sec:detailed_class_diagram}
% Begin Section
\begin{table}[H]
\centering % centers tables in page
\begin{tabular}{|>{\centering\arraybackslash}p{10cm}|} % centers text in table while specifying table width
% use immediately preceding column specifier
\hline
Client Control\\
\hline
\begin{itemize}
\item[-] clientID
\item[-] private
\end{itemize}
\\
\hline
\begin{itemize}
\item[+] constructor
\item[+] public
\end{itemize}
\\
\hline
\end{tabular}
\end{table}
%
\begin{table}[H]
\centering
\begin{tabular}{|>{\centering\arraybackslash}p{10cm}|}
\hline
Result Interface\\
\hline
\begin{itemize}
\item[-] private
\end{itemize}
\\
\hline
\begin{itemize}
\item[+] public
\end{itemize}
\\
\hline
\end{tabular}
\end{table}
%
\begin{table}[H]
\centering
\begin{tabular}{|>{\centering\arraybackslash}p{10cm}|}
\hline
Query Form\\
\hline
\begin{itemize}
\item[-] private
\end{itemize}
\\
\hline
\begin{itemize}
\item[+] public
\end{itemize}
\\
\hline
\end{tabular}
\end{table}
%
\begin{table}[H]
\centering
\begin{tabular}{|>{\centering\arraybackslash}p{10cm}|}
\hline
Server Control\\
\hline
\begin{itemize}
\item[-] expertList: Expert[3] = [expert1, expert2, expert3]
\item[-] clientID: int[] = null
\item[-] clientQuery: object[] = null
\item[-] queryResponse: object[] = null

\end{itemize}
\\
\hline
\begin{itemize}
\item[+] contructor()
\item[+] init()
\item[+] listen()
\item[+] readQuery(Query query): object
\item[+] sendResult(Object object)
\item[+] lockExpert(String expert)
\end{itemize}
\\
\hline
\end{tabular}
\end{table}
%
\begin{table}[H]
\centering
\begin{tabular}{|>{\centering\arraybackslash}p{10cm}|}
\hline
Expert 1 Control\\
\hline
%\begin{itemize}
%\item[-] private
%\end{itemize}
%\\
%\hline
\begin{itemize}
\item[+] constructor()
\item[+] init()
\item[+] update()
\item[+] search(Object query): Object[]
\item[+] addItem(Object newObject)
\item[+] removeItem(Object oldObject): Object
\end{itemize}
\\
\hline
\end{tabular}
\end{table}
%
\begin{table}[H]
\centering
\begin{tabular}{|>{\centering\arraybackslash}p{10cm}|}
\hline
Expert 1 Database\\
\hline
\begin{itemize}
\item[-] movieList: String[] = null
\item[-] paramterList: enumerate = 1..N
\item[-] parameterData1: Object[] = null
\item[-] paramterDataN: Object[] = null
\end{itemize}
\\
\hline
\begin{itemize}
\item[+] constructor()
\item[+] deconstructor()
\item[+] init()
\end{itemize}
\\
\hline
\end{tabular}
\end{table}
%
\begin{table}[H]
\centering
\begin{tabular}{|>{\centering\arraybackslash}p{10cm}|}
\hline
Expert 2 Control\\
\hline
%\begin{itemize}
%\item[-] private
%\end{itemize}
%\\
%\hline
\begin{itemize}
\item[+] constructor()
\item[+] init()
\item[+] update()
\item[+] search(Object query): Object[]
\item[+] addItem(Object newObject)
\item[+] removeItem(Object oldObject): Object
\end{itemize}
\\
\hline
\end{tabular}
\end{table}
%
\begin{table}[H]
\centering
\begin{tabular}{|>{\centering\arraybackslash}p{10cm}|}
\hline
Expert 2 Database\\
\hline
\begin{itemize}
\item[-] movieList: String[] = null
\item[-] paramterList: enumerate = 1..N
\item[-] parameterData1: Object[] = null
\item[-] paramterDataN: Object[] = null
\end{itemize}
\\
\hline
\begin{itemize}
\item[+] constructor()
\item[+] deconstructor()
\item[+] init()
\end{itemize}
\\
\hline
\end{tabular}
\end{table}
%
\begin{table}[H]
\centering
\begin{tabular}{|>{\centering\arraybackslash}p{10cm}|}
\hline
Expert 3 Control\\
\hline
%\begin{itemize}
%\item[-] private
%\end{itemize}
%\\
%\hline
\begin{itemize}
\item[+] constructor()
\item[+] init()
\item[+] update()
\item[+] search(Object query): Object[]
\item[+] addItem(Object newObject)
\item[+] removeItem(Object oldObject): Object
\end{itemize}
\\
\hline
\end{tabular}
\end{table}
%
\begin{table}[H]
\centering
\begin{tabular}{|>{\centering\arraybackslash}p{10cm}|}
\hline
Expert 3 Database\\
\hline
\begin{itemize}
\item[-] movieList: String[] = null
\item[-] paramterList: enumerate = 1..N
\item[-] parameterData1: Object[] = null
\item[-] paramterDataN: Object[] = null
\end{itemize}
\\
\hline
\begin{itemize}
\item[+] constructor()
\item[+] deconstructor()
\item[+] init()
\end{itemize}
\\
\hline
\end{tabular}
\end{table}
% End Section

\appendix
\section{Division of Labour}
\label{sec:division_of_labour}
% Begin Section
Include a Division of Labour sheet which indicates the contributions of each team member. This sheet must be signed by all team members.
% End Section

\newpage
\section*{IMPORTANT NOTES}
\begin{itemize}
	\item You do \underline{NOT} need to provide a text explanation of each diagram; the diagram should speak for itself
	\item Please document any non-standard notations that you may have used
	\begin{itemize}
		\item \emph{Rule of Thumb}: if you feel there is any doubt surrounding the meaning of your notations, document them
	\end{itemize}
	\item Some diagrams may be difficult to fit into one page
	\begin{itemize}
		\item It is OK if the text is small but please ensure that it is readable when printed
		\item If you need to break a diagram onto multiple pages, please adopt a system of doing so and throughly explain how it can be reconnected from one page to the next; if you are unsure about this, please ask me
	\end{itemize}
	\item Please submit the latest version of Deliverable 1 and Deliverable 2 with Deliverable 3
	\begin{itemize}
		\item They do not have to be a freshly printed versions; the latest marked versions are OK
	\end{itemize}
	\item If you do \underline{NOT} have a Division of Labour sheet, your deliverable will \underline{NOT} be marked
\end{itemize}

%END SECTION
\newpage
\clearpage
\appendix
\section{Division of Labour} \label{dlabour}
% Begin Section
\begin{tabular}{ |p{3cm}||p{2cm}|p{6cm}|p{1.5cm}|  }
 \hline
 \multicolumn{4}{|c|}{Contributions} \\
 \hline
 \textbf{Name}& \textbf{Student Number}& \textbf{Contribution}& \textbf{Signature}\\
 \hline
 Joshua & 1311940 & Detailed Class Diagrams &   \\ 
 &&&   \\
 &&&   \\
 &&&   \\
 \hline
 Keyur  &    &  &\\
 &&&   \\
 &&&   \\
 &&&   \\
 \hline
 Justin &&& \\
 &&&   \\
 &&&   \\
 &&&   \\
 \hline
 Bilal & & & \\
 &&&   \\
 &&&   \\
 &&&   \\
 \hline
 Shaad &  & &\\
 &&&   \\
 &&&   \\
 &&&   \\
 \hline
 Abdullah & &  &\\
 &&&   \\
 &&&   \\
 &&&   \\
 \hline
\end{tabular}

\end{document}
%------------------------------------------------------------------------------