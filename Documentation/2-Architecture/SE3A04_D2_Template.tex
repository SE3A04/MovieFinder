\documentclass[]{article}

% Imported Packages
%------------------------------------------------------------------------------
\usepackage{amssymb}
\usepackage{amstext}
\usepackage{amsthm}
\usepackage{amsmath}
\usepackage{enumerate}
\usepackage{fancyhdr}
\usepackage[margin=1in]{geometry}
\usepackage{graphicx}
\usepackage{setspace}
%------------------------------------------------------------------------------

% Header and Footer
%------------------------------------------------------------------------------
\pagestyle{plain}  
\renewcommand\headrulewidth{0.4pt}                                      
\renewcommand\footrulewidth{0.4pt}                                    
%------------------------------------------------------------------------------

% Title Details
%------------------------------------------------------------------------------
\title{Deliverable \#2 Template}
\author{SE 3A04: Software Design II -- Large System Design}
\date{}                               
%------------------------------------------------------------------------------

% Document
%------------------------------------------------------------------------------
\begin{document}

\maketitle	

\section{Introduction}
\label{sec:introduction}
% Begin Section

This section should provide an brief overview of the entire document.

\subsection{Purpose}
\label{sub:purpose}
% Begin SubSection
\begin{enumerate}[a)]
	\item Delineate the purpose of the document
	\item Specify the intended audience for the document
\end{enumerate}
% End SubSection

\subsection{System Description}
\label{sub:system_description}
% Begin SubSection
\begin{enumerate}[a)]
	\item Give a brief description of the system. This could be a paragraph or two to give some context to this document.
\end{enumerate}
% End SubSection

\subsection{Overview}
\label{sub:overview}
% Begin SubSection
\begin{enumerate}[a)]
	\item Describe what the rest of the document contains 
	\item Explain how the document is organised
\end{enumerate}
% End SubSection

% End Section

\section{Use Case Diagram}
\label{sec:use_case_diagram}
% Begin Section
\begin{enumerate}[a)]
	\item The use case for the client side of the application.\newline
	\includegraphics[scale=0.5]{UseCaseClient}
	\newpage
	\item The use case for the server side of the application.\newline
	\includegraphics[scale=0.5]{UseCaseServer}
\end{enumerate}
% End Section

\section{Analysis Class Diagram}
\label{sec:analysis_class_diagram}
% Begin Section
This section should provide an analysis class diagram for your application.
% End Section


\section{Architectural Design}
\label{sec:architectural_design}
% Begin Section\

\subsection{System Architecture}
\label{sub:system_architecture}
% Begin SubSection
The overall architecture of the system is a hybrid between a Client-Server Architecture and a Blackboard Architecture. Both of these architectures are needed to make the design more robust and simple.

In a Client-Server Architecture, there are many clients that communicate with a server. In the case of this system, this architecture makes sense because there is a client with a GUI that the users can use to make searches and have information displayed to them. There is also a server component that will contain large databases of information which would be too large to store in the client. The server can respond to a client's query by using the database of information. The reason why a Client-Server architecture is necessary is that the databases of movie information are very large and it would be too much to download considering that this system is a mobile application. This architecture allows the data to be stored on the server and only relevant parts are communicated to the client.

In the Blackboard Architecture there is a blackboard subsystem that stores data, and there is a knowledge source subsystem that stores domain specific data. In this system, the blackboard and knowledge subsystems are part of the server system. The blackboard is on the server and will contain the information about a movie, and what movie each expert thinks corresponds to this data. The knowledge source will be databases stored on the server because databases of movies are very large, much too large to store in the client system. The knowledge subsystem is the experts. The reason for using a Blackboard Architecture is that it allows for more experts to be easily added. It also makes it easy to combine the results from each individual expert in a centralized location.
% End SubSection

\subsection{Subsystems}
\label{sub:subsystems}
% Begin SubSection
\begin{description}
	\item[Blackboard] Provides information to the experts and combines the results.
	\item[Communication] Handles communication between client and server. Includes necessary encryption.
	\item[Knowledge] Contains many experts, each has a specific method of going from information about a movie, to what movie that information is about. Gets the information from the Blackboard subsystem and returns the result to it as well.
	\item[UI] Provides the user with a graphical user interface so they can interact with the client.
\end{description}
% End SubSection

% End Section
	
\section{Class Responsibility Collaboration (CRC) Cards}
\label{sec:class_responsibility_collaboration_crc_cards}
% Begin Section
This section should contain all of your CRC cards.

\begin{enumerate}[a)]
	\item Provide a CRC Card for each identified class
	\item Please use the format outlined in tutorial, i.e., 
	\begin{table}[ht]
		\centering
		\begin{tabular}{|p{5cm}|p{5cm}|}
		\hline 
		 \multicolumn{2}{|l|}{\textbf{Class Name:}} \\
		\hline
		\textbf{Responsibility:} & \textbf{Collaborators:} \\
		\hline
		\vspace{1in} & \\
		\hline
		\end{tabular}
	\end{table}
	
\end{enumerate}
% End Section

\appendix
\section{Division of Labour}
\label{sec:division_of_labour}
% Begin Section
Include a Division of Labour sheet which indicates the contributions of each team member. This sheet must be signed by all team members.
% End Section

\newpage
\section*{IMPORTANT NOTES}
\begin{itemize}
%	\item You do \underline{NOT} need to provide a text explanation of each diagram; the diagram should speak for itself
	\item Please document any non-standard notations that you may have used
	\begin{itemize}
		\item \emph{Rule of Thumb}: if you feel there is any doubt surrounding the meaning of your notations, document them
	\end{itemize}
	\item Some diagrams may be difficult to fit into one page
	\begin{itemize}
		\item It is OK if the text is small but please ensure that it is readable when printed
		\item If you need to break a diagram onto multiple pages, please adopt a system of doing so and thoroughly explain how it can be reconnected from one page to the next; if you are unsure about this, please ask about it
	\end{itemize}
	\item Please submit the latest version of Deliverable 1 with Deliverable 2
	\begin{itemize}
		\item It does not have to be a freshly printed version; the latest marked version is OK
	\end{itemize}
	\item If you do \underline{NOT} have a Division of Labour sheet, your deliverable will \underline{NOT} be marked
\end{itemize}


\end{document}
%------------------------------------------------------------------------------