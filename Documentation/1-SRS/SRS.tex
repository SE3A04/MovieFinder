\documentclass[]{article}

% Imported Packages
%------------------------------------------------------------------------------
\usepackage{amssymb}
\usepackage{amstext}
\usepackage{amsthm}
\usepackage{amsmath}
\usepackage{enumerate}
\usepackage{fancyhdr}
\usepackage[margin=1in]{geometry}
\usepackage{graphicx}
%\usepackage{extarrows}
\usepackage{setspace}
%------------------------------------------------------------------------------

% Header and Footer
%------------------------------------------------------------------------------
\pagestyle{plain}  
\renewcommand\headrulewidth{0.4pt}                                      
\renewcommand\footrulewidth{0.4pt}                                    
%------------------------------------------------------------------------------

% Title Details
%------------------------------------------------------------------------------
\title{Software Requirements Specifications}
\author{Team 1}                               
%------------------------------------------------------------------------------

\newcommand{\sname}{Movie Matcher}

% Document
%------------------------------------------------------------------------------
\begin{document}

\maketitle	

\section{Introduction}
\label{sec:introduction}
% Begin Section

This section of the SRS gives an overview of everything included in the SRS.

\subsection{Purpose}
\label{sub:purpose}
% Begin SubSection
The purpose of this document is to give a detailed description of the software. It will illustrate the features of the software, the purpose, what it will do, the requirements, any constraints, and interactions with other systems or with its environment. This document will mainly be utilized by the developers of the system, but will also be reviewed the teaching assistants.
% End SubSection

\subsection{Scope}
\label{sub:scope}
% Begin SubSection
The "\sname" is mobile application that will help the user find information relevant to a movie given some details the user knows about the movie. The application should also display information about any movies matching the user's criteria. If the user finds the movie they were looking for, they should be directed to nearby theaters that are playing the movie if it is a new movie. The user should also be given links to purchase the movie if they want. The goal of this software is to help people identify movies based on characteristics they know about the movie they are looking for. 

This software will not guarantee that it will always find the correct result, or any result at all. The software will try its best to identify nearby theaters based on its interaction with the GPS in the phone. The software will also interact over an Internet connection to perform the search.

This application is intended to be used in Canada, so it will be use Canadian English text. The input to the software is expected to Canadian English as well. Also, since the application is intended for use in Canada only, the theater location service and movie purchasing features only concern Canadian theaters and retailers.
% End SubSection

\subsection{Definitions, Acronyms, and Abbreviations}
\label{sub:definitions_acronyms_and_abbreviations}
% Begin SubSection
\begin{enumerate}[a)]
	\item Developer: The people creating the software.
	\item SRS: This document is the SRS. It stands for Software Requirements Specification. It details the requirements for the system to be.
	\item User: The person using the software.
\end{enumerate}
% End SubSection

\subsection{References}
\label{sub:references}
% Begin SubSection
None
% End SubSection

\subsection{Overview}
\label{sub:overview}
% Begin SubSection
The next section, \ref{odesc} Overall Description, gives an broad view of factors that affect the system-to-be. It gives background information which is relevant to the requirements. The section after, \ref{freq} Functional Requirements, details each of the requirements for the system. The requirements are elaborated in enough detail to allow the design of the system and allow tests to be constructed. The next section, \ref{nfreq} Non-Functional Requirements, contains qualities that the overall system should have. The last section, \ref{dlabour} Division of Labour, states who did a certain piece of work.
% End SubSection

% End Section

\section{Overall Description} \label{odesc}
\label{sec:overall_description}
% Begin Section
Movie Matcher is an application which provides a list of possible movies matching criteria that the user
gives it. To this effect the application must communicate with the user, execute internal computations and
communicate with external experts.
Given this environment, certain requirements must be defined in order for a clear product to be designed
and developed.

\subsection{Product Perspective}
\label{sub:product_perspective}
% Begin SubSection
The Movie Matcher application is an aggregate searching tool that returns possible movies matching the query made by the client (user) modified by parameters such as cast and crew, notable quotes, as well as genre and plot description. Movie Matcher will achieve this by accessing data from external and disparate experts that will be referenced against the parameters provided by the client.
% End SubSection

\subsection{Product Functions}
\label{sub:product_functions}
% Begin SubSection
Movie Matcher is intended to handle the following 3 business events:
\begin{enumerate}
	\item User enters a query
	\item User sorts and selects a result
\end{enumerate}
The Movie Matcher system will uniquely respond to all of these business events as follows:
\begin{enumerate}
\item Query its experts based on the user input parameters and process the results to return to the user a list of possible matching movies
\item The query is returned as per usual and are ordered depending on the sorting parameter
\end{enumerate}
% End SubSection

\subsection{User Characteristics}
\label{sub:user_characteristics}
% Begin SubSection
The following assumptions pertain to at least 80\% of our expected users:
\begin{enumerate}[a)]
	\item The user is able to comprehend enough of the English language to operate Movie Matcher
	\item The user is able to operate an Android OS mobile application
	\item The user has access to an operational Internet connection
\end{enumerate}
Assumptions of the users is based on aspects required of the user to operate Movie Matcher in some fashion.
% End SubSection

\subsection{Constraints}
\label{sub:constraints}
% Begin SubSection
\begin{enumerate}[a)]
	\item Application must use external connections and performance is reliant on the network that is being used
	\item The quality and format of the data of the experts is dependant on the choices of the experts
	\item The nature of the data provided by the experts may limit the possible search parameters
\end{enumerate}
% End SubSection

\subsection{Assumptions and Dependencies}
\label{sub:assumptions_and_dependencies}
% Begin SubSection
\begin{enumerate}[a)]
	\item The application will be distributed as an Android OS mobile application
	\item The application will externally communicate with the chosen experts
\end{enumerate}
% End SubSection

\subsection{Apportioning of Requirements}
\label{sub:apportioning_of_requirements}
% Begin SubSection
The social aspect of Movie Matcher has yet to be defined at this time and as such will be considered later.\\
\\
Thus when the social aspect of Movie Matcher is considered its requirements must be evaluated again.
% End SubSection

% End Section

\section{Functional Requirements} \label{freq}
\label{sec:functional_requirements}
% Begin Section

\begin{enumerate}[{BE}1.]
	
	\item User wants to find the name of a movie
	\begin{enumerate}[{VP1}.1]
		\item User
			\begin{enumerate}
				\item The product shall allow the user to search for movies based on the search criteria entered by the user.
				\item The product shall allow the user to enter a quote as search criteria for searching for movies.
				\item The product shall allow the user to choose genre/type of movies for search criteria.
				\item The product shall allow the user to enter the names of the cast for search criteria.
				\item The product shall allow the user to choose whether it wants to search box office movies only for search criteria.
			\end{enumerate}
		\item Developer
			\begin{enumerate}
				\item Developer has access to network and security.
				\item Developer has access to a ‘debug’ mode where implicit computations are visible on the console.
			\end{enumerate}
		\item Security Administrator
			\begin{enumerate}
				\item The product shall encrypt any communication between the client and the server.
			\end{enumerate}
	\end{enumerate}
	
	\item User wants to sort the results
	\begin{enumerate}[{VP2}.1]
		\item User
			\begin{enumerate}
				\item The product displays the search results with each of the three expert’s results.
				\item The product displays the best answer to the search result as well.
				\item The product allows user to sort the result of the search based on search criteria including but not limited to rating, release date.
			\end{enumerate}
		\item Developer
		\newline N/A
		\item Security Administrator
		\newline N/A
	\end{enumerate}
	\item User wants information on a result
	\begin{enumerate}[{VP3}.1]
		\item User
			\begin{enumerate}
				\item The product allows the user to select which movie to proceed with which further directs the user into a Google maps view of the location and timings of the show.
			\end{enumerate}
		\item Developer
		\newline N/A
		\item Security Administrator
		\newline N/A
	\end{enumerate}
	
	\item User wants to watch the movie
	\begin{enumerate}[{VP4}.1]
		\item User
			\begin{enumerate}
				\item The product shall display locations where the movie selected is playing.
				\item The application shall open up the phone’s navigation application.
			\end{enumerate}
		\item Developer
		\newline N/A
		\item Security Administrator
		\newline N/A
	\end{enumerate}
	
	\item Developer wants to update or exchange agents
	\begin{enumerate}[{VP5}.1]
		\item User
		\newline N/A

		\item Developer
			\begin{enumerate}
				\item The product shall allow the developer to update the experts/agents.
			\end{enumerate}
		\item Security Administrator
		\newline N/A
	\end{enumerate}
	
	\item A new agent is discovered
	\begin{enumerate}[{VP6}.1]
		\item User
		\newline N/A

		\item Developer
			\begin{enumerate}
				\item The product shall allow the developer to add and remove experts/agents.
			\end{enumerate}
		\item Security Administrator
		\newline N/A
	\end{enumerate}
	
\end{enumerate}


% End Section

\section{Non-Functional Requirements} \label{nfreq}
\label{sec:non-functional_requirements}
% Begin Section
\subsection{Look and Feel Requirements}
\label{sub:look_and_feel_requirements}
% Begin SubSection


\subsubsection{Appearance Requirements}
\label{ssub:appearance_requirements}
\newcounter{placeholder}
% Begin SubSubSection
\begin{enumerate}[{LF}1. ]
	\item The application should be visually appealing.
	\item The overall design should look familiar to other query based applications.
	\item The interface should be descriptive, telling the user what to do and how to do it.
	\item The query search and the experts should be easily visible.
	\item The results displayed should be easily readable.
	\setcounter{placeholder}{\theenumi}
\end{enumerate}
% End SubSubSection

\subsubsection{Style Requirements}
\label{ssub:style_requirements}
N/A
\begin{enumerate}
	\item hello
% Begin SubSubSection
\end{enumerate}


\subsection{Usability and Humanity Requirements}
\label{sub:usability_and_humanity_requirements}
% Begin SubSection

\subsubsection{Ease of Use Requirements}
\label{ssub:ease_of_use_requirements}
% Begin SubSubSection
\begin{enumerate}[{UH}1. ]
	\item The application should be easy to use just like an ideal query processing software (navigate, surf, search, get results, etc.)
% End SubSubSection

\subsubsection{Personalization and Internationalization Requirements}
\label{ssub:personalization_and_internationalization_requirements}

% End SubSubSection

\subsubsection{Learning Requirements}
\label{ssub:learning_requirements}
% Begin SubSubSection
	\item The overall functionality of the application should be basic, a child of 10 years or older should be able to use it.
	\item The application should be designed in a way that query-processing is not difficult.

\subsubsection{Understandability and Politeness Requirements}
\label{ssub:understandability_and_politeness_requirements}


\subsubsection{Accessibility Requirements}
\label{ssub:accessibility_requirements}
\end{enumerate}

% End SubSection

\subsection{Performance Requirements}
\label{sub:performance_requirements}
% Begin SubSection

\subsubsection{Speed and Latency Requirements}
\label{ssub:speed_and_latency_requirements}
% Begin SubSubSection
\begin{enumerate}[{PR}1. ]
	\item The application should be able to present the results of a search requested by the user within five seconds of the request being made.
	\item The application should be respond within three seconds to a user command other than a search (open, close, etc.).
% End SubSubSection

\subsubsection{Safety-Critical Requirements}
\label{ssub:safety_critical_requirements}
% Begin SubSubSection
	\item The application should project proper lighting on the screen (prevent seizures and blindness).
	\item The application must abide by all internet/software safety laws.
% End SubSubSection

\subsubsection{Precision or Accuracy Requirements}
\label{ssub:precision_or_accuracy_requirements}
	\item The software shall provide at most 10 different movies when a query search is made.
	\item The software shall perform the exact search the user requests. It shall not assume a different spelling (i.e. to correct a spelling mistake, etc.).


\subsubsection{Reliability and Availability Requirements}
\label{ssub:reliability_and_availability_requirements}
	\item The product should be usable for 24 hours per day, 365 days per year.
% End SubSubSection

\subsubsection{Robustness or Fault-Tolerance Requirements}
\label{ssub:robustness_or_fault_tolerance_requirements}

\subsubsection{Capacity Requirements}
\label{ssub:capacity_requirements}
	\item The application shall allow at most one user pewr device to operate the software.
	\item The application shall ideally search for one movie per search.
% End SubSubSection

\subsubsection{Scalability or Extensibility Requirements}
\label{ssub:scalability_or_extensibility_requirements}

% End SubSubSection

\subsubsection{Longevity Requirements}
\label{ssub:longevity_requirements}
\end{enumerate}
% End SubSubSection

% End SubSection

\subsection{Operational and Environmental Requirements}
\label{sub:operational_and_environmental_requirements}
% Begin SubSection

\subsubsection{Expected Physical Environment}
\label{ssub:expected_physical_environment}
% Begin SubSubSection
\begin{enumerate}[{OE}1. ]
	\item The product shall run on any android platform device along with computers, tablets, PC’s, and cellphones (other devices that support Android).
% End SubSubSection

\subsubsection{Expected Operational Behaviour}
\label{ssub:expected_operational_behaviour}
% Begin SubSubSection
	\item The system shall resemble a simple input/output mechanism. The user will enter a search and the software shall process the requested query and present the results.
	\item The software shall interact with the user to perform transactions.
	\item The software shall interact with the system that has it installed (memory usage, battery percentage, etc.).


\subsubsection{Requirements for Interfacing with Adjacent Systems}
\label{ssub:requirements_for_interfacing_with_adjacent_systems}


\subsubsection{Productization Requirements}
\label{ssub:productization_requirements}


\subsubsection{Release Requirements}
\label{ssub:release_requirements}
\end{enumerate}


\subsection{Maintainability and Support Requirements}
\label{sub:maintainability_and_support_requirements}
% Begin SubSection

\subsubsection{Maintenance Requirements}
\label{ssub:maintenance_requirements}
% Begin SubSubSection
\begin{enumerate}[{MS}1. ]
	\item The main software holding the components together should easily be updated when it is needed.
	\item The software should easily be modifiable just in case the user decides to add more functionality to the application.


\subsubsection{Supportability Requirements}
\label{ssub:supportability_requirements}
% Begin SubSubSection
	\item The application should be able to be used in various platforms such as Windows, Macintosh, and Android.
	\item The application should be able to be used in various devices such as a desktop computer, cellphone, laptop, etc.
% End SubSubSection

\subsubsection{Adaptability Requirements}
\label{ssub:adaptability_requirements}

\end{enumerate}

% End SubSection

\subsection{Security Requirements}
\label{sub:security_requirements}
% Begin SubSection

\subsubsection{Access Requirements}
\label{ssub:access_requirements}
% Begin SubSubSection
\begin{enumerate}[{SR}1. ]
% End SubSubSection

\subsubsection{Integrity Requirements}
\label{ssub:integrity_requirements}
	\item The application shall use encryption to protect transactions being sent from the user to the client.

% End SubSubSection

\subsubsection{Privacy Requirements}
\label{ssub:privacy_requirements}
% Begin SubSubSection
	\item The application should protect the identity of the person using the software. It shall not pass personal information (if any) to any unknown party without the explicit permission of the user.
% End SubSubSection

\subsubsection{Audit Requirements}
\label{ssub:audit_requirements}
% Begin SubSubSection


\subsubsection{Immunity Requirements}
\label{ssub:immunity_requirements}
\end{enumerate}
% End SubSubSection

% End SubSection

\subsection{Cultural and Political Requirements}
\label{sub:cultural_and_political_requirements}
% Begin SubSection

\subsubsection{Cultural Requirements}
\label{ssub:cultural_requirements}
% Begin SubSubSection
\begin{enumerate}[{CP}1. ]
	\item The application should not use symbols, text, or media which debases a race, culture, or a political landscape.
% End SubSubSection

\subsubsection{Political Requirements}
\label{ssub:political_requirements}
	\item The application shall show a disclaimer explaining any similarities that may arise to political symbol or figure is coincidental.
\end{enumerate}
% End SubSubSection

% End SubSection

\subsection{Legal Requirements}
\label{sub:legal_requirements}
% Begin SubSection

\subsubsection{Compliance Requirements}
\label{ssub:compliance_requirements}
% Begin SubSubSection
\begin{enumerate}[{LR}1. ]
	\item The application should abide by any privacy and copyright laws.
	\item The use of experts shall be done, whilst the owner permits its free usage.
% End SubSubSection

\subsubsection{Standards Requirements}
\label{ssub:standards_requirements}
\end{enumerate}


% End Section
\newpage
\appendix
\section{Division of Labour} \label{dlabour}
% Begin Section
\begin{tabular}{ |p{3cm}||p{2cm}|p{6cm}|p{1.5cm}|  }
 \hline
 \multicolumn{4}{|c|}{Contributions} \\
 \hline
 \textbf{Name}& \textbf{Student Number}& \textbf{Contribution}& \textbf{Signature}\\
 \hline
 Joshua &     &&   \\ 
 &&&   \\
 &&&   \\
 &&&   \\
 \hline
 Keyur  &    &  &\\
 &&&   \\
 &&&   \\
 &&&   \\
 \hline
 Justin &&& \\
 &&&   \\
 &&&   \\
 &&&   \\
 \hline
 Bilal & & & \\
 &&&   \\
 &&&   \\
 &&&   \\
 \hline
 Shaad &  & &\\
 &&&   \\
 &&&   \\
 &&&   \\
 \hline
 Abdullah & &  &\\
 &&&   \\
 &&&   \\
 &&&   \\
 \hline
\end{tabular}
% End Section


\end{document}
%------------------------------------------------------------------------------